\section{Einleitung}
Bei der verwendeten Maschine, handelt es sich um eine Gleichstrommaschine (GSM) mit Fremderreger- und Reihenschluss-Erregerwicklung. Die Maschine ist als Generator verschalten.\\
Jede der beiden Erregerwicklungen - wenn mit Nennstrom durchflossen - erzeugt daher \SI{50}{\percent} der Nenndurchflutung $\Theta_N$ (siehe auch Nennankerspannung $U_{A,N}$ der GSM). D.h. nur wenn beide Wicklungssysteme Nennstrom führen, stellt sich Nenndurchflutung ein.\\
Die Bemessungsdaten der Gleichstrommaschine können der folgenden Tabelle\;\ref{tab:Nenndaten_GSM} entnommen werden.
\begin{table}[htb]
\begin{center}
\begin{tabular}{ l | l }
\hline
  Bezeichnung & T-T Electric LAK 4132A\\
  Nennankerstrom $I_{A,N}$ & \SI{62}{\ampere}\\
  Nennerregerstrom $I_{E,N}$ & \SI{1.77}{\ampere}\\
  Nenndrehzahl $n_N$ & \SI{2100}{\per \minute}\\
  Nennankerspannung $U_{A,N}$ & \SI{220}{}/\SI{440}{\volt}\\
  Nennerregerspannung $U_{E,N}$ & \SI{220}{\volt}\\
  Wdst. Ankerwicklung $R_A$ & \SI{0.78}{\ohm}\\
  Wdst. Erregerwicklung $R_E$ & \SI{93}{\ohm}\\
  Wdst. Kmpoundwicklung $R_D$ & \SI{0.1}{\ohm}\\
  Kühlung & fremdbelüftet\\
\hline
\end{tabular}
\end{center}
  \caption{Maschinendaten der Gleichstrommaschine (GSM)}
  \label{tab:Nenndaten_GSM}
\end{table}\\
Die Klemmenbezeichnungen für die nachfolgenden Messschaltungen sind in nachstehender Tabelle\;\ref{tab:Klemmenbz_GSM} zusammengefasst.
\begin{table}[htb]
\begin{center}
\begin{tabular}{ l | l }
\hline
  Anker & $A1$-$A2$\\
  Reihenschluss-Erregerwicklung & $D1$-$D2$\\
  Wendepolwicklung & $B1$-$B2$\\
  Fremderregte Erregerwicklung & $F1$-$F2$\\
\hline
\end{tabular}
\end{center}
  \caption{Klemmenbezeichnungen der Gleichstrommaschine (GSM)}
  \label{tab:Klemmenbz_GSM}
\end{table}